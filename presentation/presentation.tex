%% LyX 2.0.3 created this file.  For more info, see http://www.lyx.org/.
%% Do not edit unless you really know what you are doing.
\documentclass[english]{beamer}
\usepackage{mathptmx}
\usepackage[T1]{fontenc}
\usepackage[latin9]{inputenc}
\usepackage{textcomp}
\usepackage{amsmath}
\usepackage{amssymb}

\makeatletter
%%%%%%%%%%%%%%%%%%%%%%%%%%%%%% Textclass specific LaTeX commands.
 % this default might be overridden by plain title style
 \newcommand\makebeamertitle{\frame{\maketitle}}%
 \AtBeginDocument{
   \let\origtableofcontents=\tableofcontents
   \def\tableofcontents{\@ifnextchar[{\origtableofcontents}{\gobbletableofcontents}}
   \def\gobbletableofcontents#1{\origtableofcontents}
 }
 \long\def\lyxframe#1{\@lyxframe#1\@lyxframestop}%
 \def\@lyxframe{\@ifnextchar<{\@@lyxframe}{\@@lyxframe<*>}}%
 \def\@@lyxframe<#1>{\@ifnextchar[{\@@@lyxframe<#1>}{\@@@lyxframe<#1>[]}}
 \def\@@@lyxframe<#1>[{\@ifnextchar<{\@@@@@lyxframe<#1>[}{\@@@@lyxframe<#1>[<*>][}}
 \def\@@@@@lyxframe<#1>[#2]{\@ifnextchar[{\@@@@lyxframe<#1>[#2]}{\@@@@lyxframe<#1>[#2][]}}
 \long\def\@@@@lyxframe<#1>[#2][#3]#4\@lyxframestop#5\lyxframeend{%
   \frame<#1>[#2][#3]{\frametitle{#4}#5}}
 \def\lyxframeend{} % In case there is a superfluous frame end

%%%%%%%%%%%%%%%%%%%%%%%%%%%%%% User specified LaTeX commands.
\usetheme{Warsaw}
% or ...

\setbeamercovered{transparent}
% or whatever (possibly just delete it)

\makeatother

\usepackage{babel}
\begin{document}





\title[VallidAlloy]{VallidAlloy}


\subtitle{A tool for validating Alloy specifications using test-case generation}


\author[J.Pinheiro, T.Guimar�es]{Jos� Pinheiro, pg23208 \and Tiago Guimar�es, pg22832 }


\institute{Escola de Engenharia, Departamento de Informatica\\
Universidade do Minho}


\date{Milestone 1, 2012/01/10}

\makebeamertitle


\pgfdeclareimage[height=0.5cm]{Universidade do Minho}{umlogo.jpg}

\logo{\pgfuseimage{Universidade do Minho}}



\AtBeginSubsection[]{

  \frame<beamer>{ 

    \frametitle{Outline}   

    \tableofcontents[currentsection,currentsubsection] 

  }

}



%\beamerdefaultoverlayspecification{<+->}


\lyxframeend{}\lyxframe{Outline}

\tableofcontents{}




\lyxframeend{}\section{Motivation}


\lyxframeend{}\subsection[Model Validation Through Test Case Generation]{Model Validation Through Test Case Generation}


\lyxframeend{}\lyxframe{Model Validation trough test case generation}


\framesubtitle{}
\begin{itemize}
\item Use of test case generation to validate a model.


\pause{}

\item Generate test cases from a model.


\pause{}

\item Implement tests cases.


\pause{}

\item Compare the tests cases with the model.


\pause{}

\item If any diference is found, there is a problem in the model or the
implementation.
\end{itemize}

\lyxframeend{}\lyxframe{Case-study: Git}
\begin{itemize}
\item <1->Use of Git as case study of model validation trough test case
generation.
\item <2->Implement intances of git repositories generated from an alloy
git model.
\item <3->Compare the git repositories with the model.
\item <4->The final objective is to get an a correct alloy git model.
\end{itemize}

\lyxframeend{}\lyxframe{What is Git}
\begin{itemize}
\item <1->Git is a famous distributed revision control and source code
management system.
\item <2->Complete history and full revision tracking capabilities.
\item <3->Git Data structures.
\item <4->Porcelain comands.
\item <5->Plumbling comands.
\end{itemize}

\lyxframeend{}\subsection{Alloy Git Model}


\lyxframeend{}\lyxframe{Alloy Git Model }
\begin{itemize}
\item <1->Use of an alloy git model to generate test cases.
\item <2->Ultimate goal of vallidAlloy is to improve this model.
\item <3->The alloy git model was given to us from a previous work with
this goal.
\end{itemize}

\lyxframeend{}\section{VallidAlloy}


\lyxframeend{}\subsection{Testbench}


\lyxframeend{}\lyxframe{Why a Testbench?}
\begin{itemize}
\item <1->It is the difficult to correctly see if the instances from the
alloy git model are correct and behave like git.
\item <2->Git manual is obscure.
\item <3->Generates git repositories from an alloy model.
\item <4->Compare the git repositories with the alloy model.
\end{itemize}

\lyxframeend{}\lyxframe{How it works}
\begin{itemize}
\item <1->The tesbench is made in Java to easily conect to the alloy api.
\item <2->Trough alloy api the testbench gets the solution generated from
an run in the model.
\item <3->Use of git plumbing to generate the git objects.
\item <4->Also generates the repository filysystem.
\item <5->Creates a number of diferent runs from that solution into actual
git repositories.
\end{itemize}

\lyxframeend{}\lyxframe{Work done in Milestone 1}
\begin{itemize}
\item <1->It currently replicates:\end{itemize}
\begin{enumerate}
\item <2->Filesystem from an a alloy instance;
\item <3->Git repository;
\item <3->Git Objects(blobs).
\end{enumerate}

\lyxframeend{}\subsection{Future Implementations}


\lyxframeend{}\lyxframe{Future Implementations}
\begin{itemize}
\item <1->Create all the git objects.
\item <2->Optimize workbench.
\item <3->Actualy improve the model.
\end{itemize}

\lyxframeend{}\section*{Summary}


\lyxframeend{}\lyxframe{Summary}
\begin{itemize}
\item Model validation through test case generation.
\item Use of alloy.
\item Git as case study.
\item Validate model.
\end{itemize}


\vskip0pt plus.5fill
\begin{itemize}
\item Outlook

\begin{itemize}
\item Replicate full git repository.
\item Improve alloy git model.
\end{itemize}
\end{itemize}

\lyxframeend{}

\appendix

\lyxframeend{}\section*{Appendix}


\lyxframeend{}\subsection*{For Further Reading}


\lyxframeend{}\lyxframe{[allowframebreaks]For Further Reading}

\beamertemplatebookbibitems
\begin{thebibliography}{References}
\bibitem{Daniel Jackson}Daniel Jackson. \newblockSoftware Abstractions:
Logic, Language, and Analysis\newblock The MIT Press, 2006. 1990.\beamertemplatearticlebibitems

\bibitem{Someone2002}Scott Chacon.\newblock Pro Git\emph{.} \newblock\emph{Apress
Berkely, CA, USA �2009 }.

\end{thebibliography}

\lyxframeend{}
\end{document}
